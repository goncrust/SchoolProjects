\documentclass{article}
\usepackage{amsmath}

\title{Resumos de PE}
\author{Gonçalo Rua}

\begin{document}

\maketitle

\section{Noções de Probabilidade}

\subsection*{Definições}

\begin{itemize}

    \item Uma experiência diz-se \textbf{aleatória} se:

          \begin{itemize}
              \item todos os possíveis resultados são conhecidos à partida
              \item o resulto concreto de uma experiência não é conhecido à partida
          \end{itemize}

    \item \textbf{Espaço Amostral} $(\Omega)$ - conjunto de todos os possíveis resultados de uma experiência aleatória

    \item \textbf{Acontecimento} - dada uma experiência aleatória $E$ com espaço de resultados $\Omega$, acontecimento (ou evento) é um subconjunto de $\Omega$

    \item \textbf{$\underline{A}$} - dada $E$, com espaço $\Omega$, $\underline{A}$ é o conjunto de todos os acontecimentos possíveis de $\Omega$

\end{itemize}

\subsection*{Interpretação de Laplace}

(ex) Na experiência $E_1$, a probabilidade do acontecimento $A = \{\textnormal{sair face par}\}$ é dada por $P(A) = \frac{3}{6} = 0.5$.

\subsection*{Interpretação frequencista}

A probabilidade de uma acontecimento $A$ é o limite da frequência relativa da ocorrência de $A$ numa longa sucessão de experiências realizadas sob as mesmas condições.

$$ \textnormal{(ex) }P(A) = \lim_{n\to\infty}f_n(A)=\frac{1}{2} $$

\subsection*{Interpretação subjetivista}

A probabilidade de $A$ é uma medidade pessoal (entre 0 e 1) do grau de crença sobre a ocorrência de $A$.

\subsection*{Axiomática}

\begin{itemize}
    \item Axioma 1: $P(A) \geq 0$, $\forall A \in \underline{A}$ (acontecimentos possíveis de $\Omega$)
    \item Axioma 2: $P(\Omega) = 1$
    \item Axioma 3: Para qualquer sequência de acontecimentos disjuntos 2 a 2 $A_1, \dots, A_n$ tem-se $P(\cup_{i=1}^n A_i) = \sum_{i=1}^n P(A_i)$, $n = 2, 3, \dots$
\end{itemize}

\textbf{Espaço de probabilidade} - $(\Omega, A, P)$

\textbf{Espaço mensurável de acontecimentos} - $(\Omega, \underline{A})$

\subsection*{Teoremas decorrentes}

\begin{itemize}
    \item $P(\bar{A}) = 1 - P(A)$
    \item $P(\emptyset) = 0$
    \item $A \subset B \implies P(A) \leq P(B)$ e $P(B - A) = P(B) - P(A)$
    \item $P(A) \leq 1$
    \item $P(B - A) = P(B) - P(A \cap B)$
    \item $P(A \cup B) = P(A) + P(B) - P(A \cap B)$
\end{itemize}

\subsection*{Probabilidade condicional ($P(A)$ se $B$ aconteceu)}

$$ P(A|B) = \frac{P(A \cap B)}{P(B)} $$

\subsubsection*{Axiomática}

\begin{itemize}
    \item Axioma 1: $P(A|B) \geq 0$, $\forall$ acontecimento A
    \item Axioma 2: $P(\Omega|B) = 1$
    \item Axioma 3: Para acontecimentos disjuntos $A_1, \dots, A_n$, $P(\cup_{i=1}^n A_i|B) = \sum_{i=1}^n P(A_i|B)$, $n = 1, 2, \dots$
\end{itemize}

\subsubsection*{Teoremas decorrentes}

\begin{itemize}
    \item $P(\bar{A}|B) = 1 - P(A|B)$
    \item $P(\emptyset|B) = 0$
    \item $P(A|B) \leq 1$
    \item $A_1 \subset A_2 \implies P(A_1|B)$, $P(A_2-A_1|B) = P(A_2|B) - P(A_1|B)$
    \item $P(A_2 - A_1|B) = P(A_2|B) - P(A_2 \cap A_1|B)$
    \item $P(A_1 \cup A_2|B) = P(A_1|B) + P(A_2|B) - P(A_1 \cap A_2|B)$
\end{itemize}

\subsubsection*{Teorema da probabilidade composta}

$$ P(\cap_{i=1}^n A_i) = P(A_1)P(A_2|A_1)\dots P(A_n|A_1 \cap A_2 \cap \dots \cap A_{n-1}) $$

\subsubsection*{Teorema da probabilidade total}

\emph{Partição} do espaço de resultados $\Omega$:

\begin{itemize}
    \item $A_i \cap A_j = \emptyset$, $\forall i \neq j = 1, \dots, n$
    \item $\cup_{i=1}^n A_i = \Omega$
\end{itemize}

Se $B$ um acontecimento de um espaço de resultados $\Omega$ e $A_1, \dots, A_n$ uma partição de $\Omega$:
$$ P(B) = \sum_{i=1}^n P(A_i)P(B|A_i) $$

\subsubsection*{Teorema de Bayes}

Se $B$ um acontecimento de um espaço de resultados $\Omega$, $P(B) > 0$ e $A_1, \dots, A_n$ uma partição de $\Omega$, $\forall i = 1, \dots, n$:
$$ P(A_i|B) = \frac{P(A_i \cap B)}{P(B)} = \frac{P(A_i)P(B|A_i)}{\sum_{j=1}^{n} P(A_j)P(B|A_j)} $$

\subsubsection*{Acontecimentos independentes}

\end{document}