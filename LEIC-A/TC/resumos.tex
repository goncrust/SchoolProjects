\documentclass{article}
\usepackage[a4paper, total={6in, 8in}]{geometry}
\usepackage{textcomp}
\usepackage{gensymb}
\usepackage{amsfonts}
\usepackage{amsmath}

\title{Resumos de TC}
\author{Gonçalo Rua}
\date{2023}

\begin{document}

\maketitle

\section{Conceitos Básicos}

\subsection{Definições}

\begin{itemize}
    \item Alfabeto ($\Sigma$) - conjunto finito não-vazio (de símbolos)
    \item Palavra - sequência finita de elementos de $\Sigma$
    \item $\Sigma^*$ - conjunto de todas as palavras com símbolos de $\Sigma$
    \item Palavra vazia ($\epsilon$) - pertence a $\Sigma^*$ para todo o alfabeto $\Sigma$
    \item Linguagem ($L$) - qualquer conjunto $L \subset \Sigma^*$
\end{itemize}

\subsubsection{Nota}

Num alfabeto $\Sigma$ com $n$ elementos, o número de palavras de tamanho $k$ é $n^k$.

\subsection{Operações sobre palavras}

Sendo $\omega$ a palavra 1101 e $\sigma$ a palavra 010:

\begin{itemize}
    \item Reversão ($\omega^R$) - 1011
    \item Concatenação ($\omega.\sigma$) - 1101010 ($\omega.\epsilon = \epsilon.\omega = \omega$)
    \item $\omega^n = \begin{cases} \epsilon$, $&$ se $n = 0 \\ \omega.\omega^n-1$, $&$ c.c. $\end{cases}$
    \item Comprimento da palavra $\omega$ ($|\omega|$) - 4
    \item $n$-ésimo símbolo da palavra $\omega$ ($\omega_n$)
\end{itemize}

\subsection{Operações sobre linguagens}

\begin{itemize}
    \item Linguagem complementar de $L$ ($\bar{L}$) - $\Sigma^*\backslash L$
    \item Conjunto de toas as linguages sobre $\Sigma$ ($\mathcal{L}^\Sigma$)
    \item Concatenação ($L_1.L_2$) - $\{uv : u \in L_1, v \in L_2\}$, assumindo que $L_1,L_2 \in \mathcal{L}^\Sigma$
    \item Fecho de Kleene da linguagem $L$ ($L^*$) - $\{u_1.u_2.\dots.u_n : n \in \mathbb{N}_0, u_1, u_2,\dots,u_n \in L\}$
\end{itemize}

\section{Autómatos}

\subsection{Autómatos Finitos Determinísticos (AFD)}

Um AFD é um conjunto ($\Sigma, \mathbf{Q}, q_{in}, F, \delta$):

\begin{itemize}
    \item $\Sigma$ - alfabeto
    \item $\mathbf{Q}$ - conjunto finito não vazio de estados
    \item $q_{in} \in \mathbf{Q}$ - estado inicial
    \item $F \subset \mathbf{Q}$ - conjunto de estados finais
    \item $\delta : \mathbf{Q} \times \Sigma \to \mathbf{Q}$ - função de transição
\end{itemize}

\noindent Cada AFD define uma linguagem sobre o seu alfabeto $\Sigma$.

\subsubsection{Autómatos totais}

Um autómato é \emph{total} se a função de transição em cada estado estiver definida para todas as letras.
Caso não seja total podemos convertê-lo em total:
\begin{enumerate}
    \item adicionar um estado não final $q'$
    \item estender a função de transição tal que $\delta(q,a) = q'$ para todo o par $(q,a) \in \mathbf{Q} \times \Sigma$ que a função de transição não fosse definida
    \item definir $\delta(q', a) = q'$, para todo o $a \in \Sigma$
\end{enumerate}

\subsubsection{Função de transição estendida}

$$\delta^*(q,w) = \begin{cases}
        q,                        & \text{se } w = \epsilon \\
        \delta^*(\delta(q,a),w'), & \text{se } w = a.w'
    \end{cases}$$

\subsubsection{Linguagem reconhecida e linguagem regular}

Uma palavra $w \in \Sigma^*$ é aceite por um AFD/AFND se $\delta^*(q_{in}, w) \in F$.

\noindent O conjunto de palavras aceites por um AFD/AFND chama-se \emph{linguagem reconhecida} por esse AFD/AFND:
$$L(D) = \{w \in \Sigma^* : \delta^*(q_{in}, w) \in F \}$$

\noindent Uma linguagem $L$ diz-se \emph{regular} se existe um AFD $D$ tal que $L(D) = L$. Denota-se por $\mathcal{REG}^\Sigma$ o conjunto de todas as linguagens regulares com alfabeto $\Sigma$.

\subsubsection{Propriedades de fecho}

TODO

\subsection{Autómato Finito Não Determinístico (AFND)}

Um AFND é um conjunto $(\Sigma, \mathbf{Q}, q_{in}, F, \delta)$:

\begin{itemize}
    \item $\Sigma$ - alfabeto
    \item $\mathbf{Q}$ - conjunto finito de estados
    \item $q_{in} \in \mathbf{Q}$ - estado inicial
    \item $F \subset \mathbf{Q}$ - conjunto de estados finais
    \item $\delta : Q \times \Sigma \to \wp(Q)$ - função de transição
\end{itemize}

\subsubsection{Conjunto das partes (de $S$)}

$\wp(S)$ representa o conjunto dos subconjuntos do conjunto $S$.

$$ \text{(ex) } \wp(\{0,1\}) = \{\emptyset, \{0\}, \{1\}, \{0,1\}\} $$

\noindent O tamanho do conjunto $\wp(S)$ é $2^n$, sendo $n$ o tamanho do conjunto $S$.

\subsubsection{AFND com trasições-$\epsilon$}

Um $\text{AFND}^\epsilon$ é um AFND em que a sua função de transição tem domínio $Q \times (\Sigma \times \{\epsilon\})$.

\subsubsection{Fecho-$\epsilon$}

O fecho-$\epsilon$ do estado $q \in \mathbf{Q}$ é o conjunto $q^\epsilon \subset \mathbf{Q}$ tal que:

\begin{itemize}
    \item $q \in q^\epsilon$
    \item $q' \in q^\epsilon \implies \delta(q', \epsilon) \subset q^\epsilon$
\end{itemize}

\subsubsection{Função de transição estendida}

$$ \delta^*(q, w) = \begin{cases}
        q^\epsilon                                                              & \text{se } w = \epsilon \\
        \cup_{q' \in q^\epsilon}(\cup_{q'' \in \delta(q', a)}\delta^*(q'', w')) & \text{se } w = a.w'
    \end{cases} $$

\subsection{Conversão de AFNDs em AFDs}

Todas as linguagens reconhecidas por AFNDs são regulares (existe um AFD que as reconhece).

\subsubsection{Remoção de transições-$\epsilon$}

Dado um AFND $A = (\Sigma, \mathbf{Q}, q_in, F, \delta)$, temos que o AFND $A' = (\Sigma, \mathbf(Q), q_in, F', \delta')$ é-lhe equivalente se:

\begin{enumerate}
    \item Se podermos alcançar um estado final através de um movimento-$\epsilon$, podemos considerar esse estado como sendo final: $$ F' = \{q \in Q : q^\epsilon \cap F \neq \emptyset\} $$
    \item Para cada estado $q \in \mathbf{Q}$ vamos ver que estados conseguimos alcançar usando apenas a letra $a \in \Sigma$. O conjunto de estados que conseguimos alcançar só com $a$ corresponde ao resultado de aplicar $a$ a todos os estados em $q^\epsilon$ e depois tirar o fecho-$\epsilon$ do resultado: $$ \delta' : \mathbf{Q} \times \Sigma \to \wp(\mathbf{Q}) \text{ é tal que } \delta'(q,a) = \bigcup_{q' \in q^\epsilon}(\bigcup_{q'' \in \delta(q', a)}q''^\epsilon) \text{ para cada } q \in \mathbf{Q} \text{ e } a \in Sigma $$
\end{enumerate}

\subsubsection{Passar de AFND para AFD}

Depois de removermos as transições-$\epsilon$ podemos passar o AFND para AFD. \\
Dado um AFND $A = (\Sigma, \mathbf{Q}, q_in, F, \delta)$, temos que o AFD $D = (\Sigma, \wp(\mathbf{Q}), q_in, F', \delta')$ é-lhe equivalente se:

\begin{enumerate}
    \item $F' = \{\mathbf{C} \subset \mathbf{Q} : \mathbf{C} \cap F \neq \emptyset\}$
    \item $\delta' : \wp(\mathbf{Q}) \times \Sigma \to \wp(\mathbf{Q}) \text{ é tal que } \delta'(\mathbf{C}, a) = \bigcup_{q \in \mathbf{C}}\delta(q,a) \text{ para cada } \mathbf{C} \subset \mathbf{Q} \text{ e } a \in \Sigma$
\end{enumerate}

\end{document}