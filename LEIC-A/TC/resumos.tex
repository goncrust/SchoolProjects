
\documentclass{article}
\usepackage[a4paper, total={6in, 8in}]{geometry}
\usepackage{amsfonts}
\usepackage{amsmath}

\title{Resumos de TC}
\author{Gonçalo Rua}
\date{2023}

\begin{document}

\maketitle

\section{Conceitos Básicos}

\subsection{Definições}

\begin{itemize}
    \item Alfabeto ($\Sigma$) - conjunto finito não-vazio (de símbolos)
    \item Palavra - sequência finita de elementos de $\Sigma$
    \item $\Sigma^*$ - conjunto de todas as palavras com símbolos de $\Sigma$
    \item Palavra vazia ($\epsilon$) - pertence a $\Sigma^*$ para todo o alfabeto $\Sigma$
    \item Linguagem ($L$) - qualquer conjunto $L \subset \Sigma^*$
\end{itemize}

\subsubsection{Nota}

Num alfabeto $\Sigma$ com $n$ elementos, o número de palavras de tamanho $k$ é $n^k$.

\subsection{Operações sobre palavras}

Sendo $\omega$ a palavra 1101 e $\sigma$ a palavra 010:

\begin{itemize}
    \item Reversão ($\omega^R$) - 1011
    \item Concatenação ($\omega.\sigma$) - 1101010 ($\omega.\epsilon = \epsilon.\omega = \omega$)
    \item $\omega^n = \begin{cases} \epsilon$, $&$ se $n = 0 \\ \omega.\omega^n-1$, $&$ c.c. $\end{cases}$
    \item Comprimento da palavra $\omega$ ($|\omega|$) - 4
    \item $n$-ésimo símbolo da palavra $\omega$ ($\omega_n$)
\end{itemize}

\subsection{Operações sobre linguagens}

\begin{itemize}
    \item Linguagem complementar de $L$ ($\bar{L}$) - $\Sigma^*\backslash L$
    \item Conjunto de toas as linguages sobre $\Sigma$ ($\mathcal{L}^\Sigma$)
    \item Concatenação ($L_1.L_2$) - $\{uv : u \in L_1, v \in L_2\}$, assumindo que $L_1,L_2 \in \mathcal{L}^\Sigma$
    \item Fecho de Kleene da linguagem $L$ ($L^*$) - $\{u_1.u_2.\dots.u_n : n \in \mathbb{N}_0, u_1, u_2,\dots,u_n \in L\}$
\end{itemize}

\end{document}